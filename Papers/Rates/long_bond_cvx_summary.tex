\documentclass[12pt]{article}
\usepackage{amsmath, amssymb, amsfonts}
\usepackage{geometry}
\usepackage{mathtools}
\usepackage{physics}
\usepackage{bm}
\geometry{margin=1in}

\title{Beyond Convexity: A Mathematical Summary and Derivation}
\author{Summary and Derivations Prepared by ChatGPT}
\date{}

\begin{document}

\maketitle

\begin{abstract}
This document presents a complete, mathematically detailed summary of 
Jessica James' research note ``Beyond Convexity'' (Commerzbank, 2023).
We derive the expressions behind bond duration, convexity, and the
higher--order Taylor terms (third and fourth derivatives of price with
respect to yield). These higher terms become essential for super-long
bonds (50--100 year maturities), especially in low-yield environments
(0--1\%). We explain the behaviour of these expansions, the breakdown of
convexity-based intuition, and the practical implications for hedging
long-bond exposures using long-term swap rates. All figures from the 
original paper are omitted; instead, their qualitative information is
described in the text.
\end{abstract}

\tableofcontents

\newpage

%%%%%%%%%%%%%%%%%%%%%%%%%%%%%%%%%%%%%%%%%%%%%%%%%%%%%%%%%%%%%%%
\section{Introduction}
%%%%%%%%%%%%%%%%%%%%%%%%%%%%%%%%%%%%%%%%%%%%%%%%%%%%%%%%%%%%%%%

Traditionally, the sensitivity of a bond's price to yield changes is
summarised using two quantities:
\begin{itemize}
    \item Duration (first derivative of price with respect to yield),
    \item Convexity (second derivative).
\end{itemize}

For ordinary maturities (up to roughly 30 years), these two terms provide
an excellent approximation of the price--yield relationship for moderate
yield changes (e.g.\ $\pm 50$--$100$bp).  
However, in the case of very long bonds (50y--100y), 
and in particular when yields are extremely low (0--1\%),
the price--yield curve becomes \emph{so curved} that
duration + convexity no longer give correct results.

This paper derives the full Taylor expansion of bond returns
\[
R = \frac{\Delta PV}{PV}
\]
and shows that the \emph{third} and \emph{fourth} derivatives become 
quantitatively important for 100-year bonds.

%%%%%%%%%%%%%%%%%%%%%%%%%%%%%%%%%%%%%%%%%%%%%%%%%%%%%%%%%%%%%%%
\section{Present Value Representation}
%%%%%%%%%%%%%%%%%%%%%%%%%%%%%%%%%%%%%%%%%%%%%%%%%%%%%%%%%%%%%%%

Let a fixed-coupon bond have:
\begin{itemize}
\item Coupon rate $c$ (annualised, expressed as a decimal, e.g.\ $c = 0.02$),
\item Yield to maturity $y$,
\item Maturity $n$ (in years),
\item Unit notional.
\end{itemize}

The bond's present value is
\begin{equation}
PV(c,y,n)
= \sum_{k=1}^{n} \frac{c}{(1+y)^k}
  + \frac{1 + c}{(1+y)^n}.
\label{eq:PV_basic}
\end{equation}
This is a finite geometric series.

Define
\begin{equation}
V = \frac{1}{1+y}.
\label{eq:V_def}
\end{equation}
Then the PV can be rewritten in closed form:
\begin{equation}
PV
= c V \frac{1 - V^{n}}{1 - V} + V^n.
\label{eq:PV_geometric}
\end{equation}

Equation \eqref{eq:PV_geometric} is the key starting point for computing
derivatives with respect to yield.  
It expresses the PV entirely in terms of the variable $V(y)$.

%%%%%%%%%%%%%%%%%%%%%%%%%%%%%%%%%%%%%%%%%%%%%%%%%%%%%%%%%%%%%%%
\section{Bond Return and Taylor Expansion}
%%%%%%%%%%%%%%%%%%%%%%%%%%%%%%%%%%%%%%%%%%%%%%%%%%%%%%%%%%%%%%%

Define the (relative) bond return for a yield move $\Delta y$:
\[
R = \frac{\Delta PV}{PV}.
\]

We expand $PV(y + \Delta y)$ in a Taylor series around $y$.
The expansion for $R$ becomes:
\begin{align}
R
&=
\frac{1}{PV} \frac{dPV}{dy} \, \Delta y
+ \frac{1}{2} \frac{1}{PV} \frac{d^2 PV}{dy^2} (\Delta y)^2
+ \frac{1}{6} \frac{1}{PV} \frac{d^3 PV}{dy^3} (\Delta y)^3  \nonumber \\
&\qquad
+ \frac{1}{24} \frac{1}{PV} \frac{d^4 PV}{dy^4} (\Delta y)^4
+ \cdots
\label{eq:bond_return_taylor}
\end{align}

The first term corresponds to \emph{duration},
the second to \emph{convexity},
and the remaining terms are the higher derivatives.
For long-tenor, low-yield bonds these higher derivatives are not negligible.

%%%%%%%%%%%%%%%%%%%%%%%%%%%%%%%%%%%%%%%%%%%%%%%%%%%%%%%%%%%%%%%
\section{Derivatives of PV with respect to Yield}
%%%%%%%%%%%%%%%%%%%%%%%%%%%%%%%%%%%%%%%%%%%%%%%%%%%%%%%%%%%%%%%

All derivatives follow from the chain rule applied to $V(y) = (1+y)^{-1}$.
We list the components:

\begin{align*}
\frac{dV}{dy} &= - (1+y)^{-2} = -V^{2}, \\
\frac{d^2V}{dy^2} &= 2 (1+y)^{-3} = 2 V^{3}, \\
\frac{d^3V}{dy^3} &= -6 (1+y)^{-4} = -6 V^{4}, \\
\frac{d^4V}{dy^4} &= 24 (1+y)^{-5} = 24 V^{5}.
\end{align*}

The PV formula \eqref{eq:PV_geometric} must be differentiated term-by-term.
To keep the exposition readable, we record only the final derivative
expressions, leaving intermediate algebraic manipulations aside.
(These can also be automated in symbolic algebra systems.)

%%%%%%%%%%%%%%%%%%%%%%%%%%%%%%%%%%%%%%%%%%%%%%%%%%%%%%%%%%%%%%%
\subsection{First Derivative (Duration Term)}
%%%%%%%%%%%%%%%%%%%%%%%%%%%%%%%%%%%%%%%%%%%%%%%%%%%%%%%%%%%%%%%

We define \emph{modified duration} $D$ through:
\[
D = -\frac{1}{PV} \frac{dPV}{dy}.
\]

Differentiating \eqref{eq:PV_geometric}, one obtains:
\begin{equation}
\frac{dPV}{dy}
= -c V^2 \frac{1 - V^{n}}{1 - V}
- c V \frac{d}{dy}\!\left(\frac{1 - V^{n}}{1 - V}\right)
- n V^{n+1}.
\label{eq:dPVdy}
\end{equation}

The middle term involves derivatives of quotient
$\frac{1 - V^n}{1 - V}$ and introduces $V^{n+k}$ terms.
For long maturities, $V^n$ is very small unless yields are very low.

%%%%%%%%%%%%%%%%%%%%%%%%%%%%%%%%%%%%%%%%%%%%%%%%%%%%%%%%%%%%%%%
\subsection{Second Derivative (Convexity Term)}
%%%%%%%%%%%%%%%%%%%%%%%%%%%%%%%%%%%%%%%%%%%%%%%%%%%%%%%%%%%%%%%

Convexity is
\[
C = \frac{1}{PV} \frac{d^2 PV}{dy^2}.
\]

The second derivative follows by differentiating \eqref{eq:dPVdy}.
The resulting expression contains combinations of
$V^{2}$, $V^{3}$, $V^{n+2}$, $V^{n+3}$, etc.
The magnitude increases dramatically as $y \to 0$.

%%%%%%%%%%%%%%%%%%%%%%%%%%%%%%%%%%%%%%%%%%%%%%%%%%%%%%%%%%%%%%%
\subsection{Third Derivative}
%%%%%%%%%%%%%%%%%%%%%%%%%%%%%%%%%%%%%%%%%%%%%%%%%%%%%%%%%%%%%%%

The third derivative is
\[
T_3 = \frac{1}{PV}\frac{d^3PV}{dy^3}.
\]

The expression is long but qualitatively follows:
\[
\frac{d^3 PV}{dy^3}
= \alpha_1 V^4
+ \alpha_2 V^5
+ \alpha_3 n V^{n+3}
+ \alpha_4 n^2 V^{n+3}
+ \alpha_5 n^3 V^{n+3},
\]
with coefficients $\alpha_i$ depending on coupon $c$.
The key observation is that as $V = (1+y)^{-1}$ approaches $1$,
all $V^k$ terms decay slowly, and the pre-factors involving $n$ 
cause enormous magnification.

%%%%%%%%%%%%%%%%%%%%%%%%%%%%%%%%%%%%%%%%%%%%%%%%%%%%%%%%%%%%%%%
\subsection{Fourth Derivative}
%%%%%%%%%%%%%%%%%%%%%%%%%%%%%%%%%%%%%%%%%%%%%%%%%%%%%%%%%%%%%%%

Similarly,
\[
T_4 = \frac{1}{PV}\frac{d^4 PV}{dy^4},
\]
and
\[
\frac{d^4PV}{dy^4}
= \beta_1 V^5
+ \beta_2 V^6
+ \beta_3 n V^{n+4}
+ \beta_4 n^2 V^{n+4}
+ \beta_5 n^3 V^{n+4}
+ \beta_6 n^4 V^{n+4}.
\]

For a 100y bond at 1\% yield,
the term $n^4 V^{n+4}$ becomes large enough that 
$(\Delta y)^4$ contributions 
are comparable to the $(\Delta y)^2$ convexity term for
moves of a few hundred basis points.

%%%%%%%%%%%%%%%%%%%%%%%%%%%%%%%%%%%%%%%%%%%%%%%%%%%%%%%%%%%%%%%
\section{Asymptotic Behaviour as yield Approaches Zero}
%%%%%%%%%%%%%%%%%%%%%%%%%%%%%%%%%%%%%%%%%%%%%%%%%%%%%%%%%%%%%%%

When yields approach zero,
\[
V = \frac{1}{1+y} \approx 1 - y + y^2 - \cdots,
\]
and long-tenor cashflows become almost undiscounted.

For the derivatives:
\[
\frac{dPV}{dy} = O\left(\frac{n}{y}\right), \qquad
\frac{d^2PV}{dy^2} = O\left(\frac{n}{y^2}\right),
\]
\[
\frac{d^3PV}{dy^3} = O\left(\frac{n}{y^3}\right), \qquad
\frac{d^4PV}{dy^4} = O\left(\frac{n}{y^4}\right).
\]

Thus, the Taylor terms scale like:
\[
(\Delta y)^k \cdot \frac{1}{y^k},
\]
so if $y=1\%$ and $\Delta y = 2\%$,  
the factors $(\Delta y / y)^k$ become huge (2, 4, 8, 16).

This explains why convexity (quadratic term) alone produces
qualitatively wrong behaviour for 100y bonds in a low-yield environment.

%%%%%%%%%%%%%%%%%%%%%%%%%%%%%%%%%%%%%%%%%%%%%%%%%%%%%%%%%%%%%%%
\section{Why Convexity Fails for Super-Long Bonds}
%%%%%%%%%%%%%%%%%%%%%%%%%%%%%%%%%%%%%%%%%%%%%%%%%%%%%%%%%%%%%%%

The convexity contribution to return is
\[
\frac12 C (\Delta y)^2,
\]
which is \emph{always positive}.  
Thus, for sufficiently large $\Delta y > 0$, the quadratic term dominates
the linear duration term and predicts that price increases with yield,
which is impossible (bond value must fall).

In the figures of the original paper (described):

\begin{itemize}
\item For a \textbf{30y bond}, the true return is well approximated by
      duration + convexity for $\pm 1\%$ yield change.

\item For a \textbf{100y bond}, convexity alone bends the approximation
      upwards for large positive $\Delta y$, giving the illusion that
      bond price may increase for sufficiently large rate rises.

\item The \textbf{third term} (cubic) corrects this behaviour,
      pulling the curve back down for positive $\Delta y$,
      and upward for negative $\Delta y$ (asymmetric shape).

\item The \textbf{fourth term} (quartic) makes the approximation match
      the true return almost perfectly even for $\pm 2\%$ shifts.
\end{itemize}

%%%%%%%%%%%%%%%%%%%%%%%%%%%%%%%%%%%%%%%%%%%%%%%%%%%%%%%%%%%%%%%
\section{Interpretation of Figures (Textual Explanation)}
%%%%%%%%%%%%%%%%%%%%%%%%%%%%%%%%%%%%%%%%%%%%%%%%%%%%%%%%%%%%%%%

\subsection*{Figure: Price–Yield Return Curves (30y vs.\ 100y)}
The original chart shows that:
\begin{itemize}
\item 30y bonds have a smooth, modestly curved price–yield relationship.
\item 100y bonds have an extremely curved relation—returns are much more
      sensitive to even small yield changes.
\end{itemize}


\subsection*{Figure: Taylor Approximations vs. True Return}
\begin{itemize}
\item For 30y bonds: duration + convexity is nearly exact.
\item For 100y bonds:
  \begin{itemize}
    \item Duration only: huge error.
    \item Duration + convexity: wrong curvature for large $\Delta y$.
    \item Add third term: major improvement.
    \item Add fourth term: nearly perfect.
  \end{itemize}
\end{itemize}

\subsection*{Residuals Plots}
Residual = true return minus approximation.
\begin{itemize}
\item Convexity produces quadratic-shaped residuals.
\item Third term corrects one side but over-corrects the other.
\item Fourth term removes almost all remaining structure.
\end{itemize}

%%%%%%%%%%%%%%%%%%%%%%%%%%%%%%%%%%%%%%%%%%%%%%%%%%%%%%%%%%%%%%%
\section{Real-World Case Studies}
%%%%%%%%%%%%%%%%%%%%%%%%%%%%%%%%%%%%%%%%%%%%%%%%%%%%%%%%%%%%%%%

Two important examples in the paper:
\begin{itemize}
\item \textbf{NRW 2119} (100-year bond)
\item \textbf{OAT 2072} (50-year French government bond)
\end{itemize}

For investors who bought these bonds at yield lows (2020--2021):
\begin{itemize}
\item The NRW 100y bond suffered losses of up to \textbf{70\%} after a
      250bp rise.
\item Duration alone drastically underestimates losses.
\item Duration + convexity still gives incorrect results.
\item Only the inclusion of third and fourth Taylor terms can replicate
      the observed returns.
\end{itemize}

%%%%%%%%%%%%%%%%%%%%%%%%%%%%%%%%%%%%%%%%%%%%%%%%%%%%%%%%%%%%%%%
\section{Hedging 100-Year Bonds with 50-Year Swaps}
%%%%%%%%%%%%%%%%%%%%%%%%%%%%%%%%%%%%%%%%%%%%%%%%%%%%%%%%%%%%%%%

The paper studies a hedge that is:
\[
\text{Hedge ratio} = \frac{\text{Duration of bond}}{\text{Duration of swap}}.
\]

For the NRW bond:
\[
D_{\text{bond}} \approx 58.8,\qquad D_{\text{swap}} \approx 55.7,
\]
yielding a hedge ratio of approximately $1.06$.

The result:
\begin{itemize}
\item Duration terms cancel by construction.
\item The hedge leaves the investor long convexity:
      positive returns for both rising and falling yields.
\item Third and fourth terms reduce this benefit for large rate rises,
      and amplify it for rate falls.
\end{itemize}

Qualitatively, the hedged 100y bond still had positive performance
during the stress period due to the long-convexity position.

%%%%%%%%%%%%%%%%%%%%%%%%%%%%%%%%%%%%%%%%%%%%%%%%%%%%%%%%%%%%%%%
\section{Conclusions}
%%%%%%%%%%%%%%%%%%%%%%%%%%%%%%%%%%%%%%%%%%%%%%%%%%%%%%%%%%%%%%%

\begin{itemize}
\item Traditional duration + convexity intuition breaks down for
      super-long bonds (50y--100y) at low yields.
\item The Taylor expansion of bond returns requires at least
      up to the \textbf{fourth} derivative for accurate results.
\item Convexity alone eventually predicts the wrong sign for returns
      under large positive rate moves.
\item Third and fourth derivatives correct this behaviour.
\item Hedging long bonds with long-term swaps produces positive convexity
      exposure, but requires awareness of higher-order risks.
\end{itemize}

\end{document}
