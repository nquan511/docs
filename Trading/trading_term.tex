% Rewritten Fixed Income Trading Language Cheat Sheet
% Improved flow and explanation; all original content preserved and expanded

\documentclass{article}
\usepackage{amsmath}
\usepackage{booktabs}
\usepackage{geometry}
\usepackage{tikz}
\usepackage{pgfplots}
\pgfplotsset{compat=1.17}
\geometry{margin=1in}

\title{Fixed Income Trading Language Cheat Sheet (Rewritten for Clarity and Flow)}

\begin{document}
\maketitle

\section{Core Abbreviations and Market Instruments}
This section lists the most common instruments and abbreviations used in global fixed income trading. These are essential for reading trader commentary and market colour.

\begin{itemize}
  \item \textbf{UB} = Ultra Bond (long-dated US Treasury future).
  \item \textbf{RX} = Euro Bund future (German 10y government bond).
  \item \textbf{German Bond Futures:}
    \begin{itemize}
        \item \textbf{Bund} = 8.5--10.5y maturity bucket.
        \item \textbf{Bobl} = 4.5--5.5y maturity bucket.
        \item \textbf{Schatz} = 1.75--2.25y maturity bucket.
    \end{itemize}
  \item \textbf{UST Classification:}
    \begin{itemize}
        \item \textbf{Bills}: $<$1y, issued at discount (e.g., \texttt{ZBC}).
        \item \textbf{Notes}: 2y--10y.
        \item \textbf{Bonds}: $>$10y.
    \end{itemize}
  \item \textbf{30s50s}: Yield spread between 50-year bond and 30-year bond.
\end{itemize}

\section{Price Movements and Trading Language}
Understanding how traders talk about price/yield dynamics is crucial. Bonds move inversely to yields:

\begin{itemize}
  \item Price \(\uparrow\) $\Rightarrow$ Yield \(\downarrow\): known as a \emph{rally}.
  \item Price \(\downarrow\) $\Rightarrow$ Yield \(\uparrow\): known as a \emph{sell-off}.
  \item \textbf{Cheapening} = Prices fall, yields rise.
  \item \textbf{Richening} = Prices rise, yields fall.
  \item \textbf{Market paid} = Traders are paying fixed (swap yields rise, swap prices fall).
\end{itemize}

These terms summarise both rate-direction and relative value dynamics.

\section{Long vs Short Positions}

\subsection{Bonds}
\begin{itemize}
  \item \textbf{Long bond}: Receive coupons \(\Rightarrow\) benefits from \textbf{falling rates} (bond prices rise).
  \item \textbf{Short bond}: Pay coupons synthetically \(\Rightarrow\) benefits from \textbf{rising rates} (bond prices fall).
\end{itemize}

\subsection{Interest Rate Swaps}
A swap position mirrors the duration profile of a bond position:

\begin{itemize}
  \item \textbf{Buy a swap} = Pay fixed, receive floating.
    \begin{itemize}
        \item Duration exposure similar to being \textbf{short a bond}.
        \item \textbf{Positive rate delta}: profits if rates rise.
    \end{itemize}
  \item \textbf{Sell a swap} = Receive fixed, pay floating.
    \begin{itemize}
        \item Duration exposure similar to being \textbf{long a bond}.
        \item \textbf{Negative rate delta}: profits if rates fall.
    \end{itemize}
\end{itemize}

\noindent Trader shorthand in swaps:
\[
\text{``Mine''} = \text{Pay fixed}, \qquad 
\text{``Yours''} = \text{Receive fixed}.
\]

If a tenor is \textbf{cheap}, traders typically \emph{receive fixed} (go long duration). If a tenor is \textbf{rich}, they \emph{pay fixed} (go short duration).

\subsection{Swap Spreads}
Swap spreads measure how expensive swap rates are relative to government bond yields of the same maturity:

\[
\text{Swap Spread} = \text{Swap Rate} - \text{Government Yield}.
\]

\begin{itemize}
  \item Measures credit/liquidity risk premium of swaps vs true risk-free rates.
  \item \textbf{Widening swap spreads}: swap rates rising relative to govies.
    \begin{itemize}
      \item Can indicate rising counterparty or liquidity risk.
      \item Often associated with \textbf{risk-off} flows (bullish for govies).
    \end{itemize}
  \item \textbf{Tightening swap spreads}: swap rates falling relative to govies.
    \begin{itemize}
      \item Common during QE or central bank balance sheet expansion.
    \end{itemize}
  \item \textbf{Long swap spread}: Receive fixed in swap \textbf{and} short the corresponding government bond.
  \item \textbf{Short swap spread}: Pay fixed in swap \textbf{and} long the government bond.
\end{itemize}


\section{Execution Slang}
These are extremely common phrases used across rates trading desks:

\begin{center}
\begin{tabular}{@{}ll@{}}
\toprule
\textbf{Phrase} & \textbf{Meaning} \\
\midrule
``Hit my bid'' / ``I got hit'' & I bought (someone sold to me) \\
``Lift my offer'' / ``I got lifted'' & I sold (someone bought from me) \\
``Mine'' & I buy \\
``Yours'' & I sell \\
``Take'' & I buy \\
``Give'' & I sell \\
\bottomrule
\end{tabular}
\end{center}

These terms help compress trade negotiation into one or two words.


\section{Yield Curve Dynamics and Duration Intuition}
Yield curve moves are not just directional—they also imply relative duration risk across maturities. Traders classify moves using four standard scenarios.

\subsection*{Types of Curve Shifts}

\begin{itemize}
  \item \textbf{Bull Steepener:}
    \begin{itemize}
      \item Short-end yields fall more than long-end yields.
      \item Typically reflects expectations of central bank easing.
      \item Front-end duration benefits the most.
    \end{itemize}
  \item \textbf{Bear Flattener:}
    \begin{itemize}
      \item Long-end sells off more than short-end.
      \item Often driven by inflation or heavy long-end supply.
      \item Negative convexity concentrated in long-end duration.
    \end{itemize}
  \item \textbf{Bull Flattener:}
    \begin{itemize}
      \item Long-end rallies more than short-end.
      \item Common in flight-to-quality episodes.
      \item Great for long-duration exposure.
    \end{itemize}
  \item \textbf{Bear Steepener:}
    \begin{itemize}
      \item Short-end sells off more than long-end.
      \item Usually driven by hawkish central bank surprises.
      \item Front-end duration performs the worst.
    \end{itemize}
\end{itemize}

\subsection*{Trader Mnemonic}
\begin{itemize}
  \item \textbf{S}teepening $\Rightarrow$ \textbf{S}horter duration exposure wins.
  \item \textbf{F}lattening $\Rightarrow$ \textbf{F}atter (longer) duration exposure wins.
\end{itemize}

\end{document}