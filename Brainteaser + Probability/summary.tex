\documentclass[11pt]{article}
\usepackage{amsmath,amsfonts,amssymb,amsthm}
\usepackage{hyperref}
\usepackage{bm}
\usepackage{mathtools}
\usepackage{tikz}
\usepackage{enumitem}

\title{Probability Notes: Cards, Coins, Dice, Random Variables, and Useful Techniques}
\date{June 2025}
\author{}

\begin{document}

\maketitle
\tableofcontents
\newpage

%%%%%%%%%%%%%%%%%%%%%%%%%%%%%%%%%%%%%%%%%%%%%%%%%%%%%%%%%%%%%%
\section{Card Problems}
%%%%%%%%%%%%%%%%%%%%%%%%%%%%%%%%%%%%%%%%%%%%%%%%%%%%%%%%%%%%%%

\subsection{Drawing $n$ Cards from a Deck}
We begin with standard techniques used in drawing cards without replacement.

\subsubsection{Expected Total Value}
Let $X_i$ be the value of the $i$-th drawn card. Because expectation is linear,
\[
\mathbb{E}\big[X_1+\cdots+X_n\big]=n\,\mathbb{E}[X_1].
\]
In a standard 52-card deck with ranks $\{1,\dots,13\}$ equally represented,
\[
\mathbb{E}[X_1]=\frac{1+2+\cdots+13}{13}=7.
\]

\paragraph{Intuition.}
Even though later draws depend on earlier ones, linearity of expectation does \emph{not} require independence. That is why the formula remains simple.

\subsubsection{Including Jokers}
If two Jokers worth 100 are added (giving 54 cards),
\[
\mathbb{E}[X_1]=\frac{4}{54}(1+\cdots+13)+\frac{2}{54}\cdot 100.
\]

\subsubsection{Replacing the Minimum of the First $n{+}1$ Cards}
Suppose we draw $n+1$ cards and replace the minimum card by a new card. The expected sum becomes
\[
\mathbb{E}\Big[\sum_{i=1}^{n+1} X_i\Big] - \mathbb{E}[\min(X_1,\ldots,X_{n+1})].
\]

To compute $\mathbb{E}[M]$ where $M=\min(X_1,\ldots,X_{n+1})$ and each $X_i$ is uniform on $\{1,\dots,13\}$:
\[
\mathbb{P}(M>k)=\left(1-\frac{k}{13}\right)^{n+1}.
\]
Then
\[
\mathbb{E}[M]=\sum_{k=1}^{13}\mathbb{P}(M\ge k).
\]

\paragraph{Intuition.}
The minimum is small precisely when at least one value is small. Because for discrete variables it is easier to compute $\mathbb{P}(M>k)$, we use a cumulative expectation trick.

\subsubsection{Expected Position of the First Ace}
Let $X$ be the number of cards revealed before the first Ace appears. Let the 48 non-Ace cards be ``ordinary’’ cards.

For each ordinary card $i$, define indicator:
\[
I_i=\mathbf{1}\{\text{card $i$ appears before any Ace}\}.
\]
Then
\[
X=1+\sum_{i=1}^{48} I_i.
\]

Because all orderings of the deck are equally likely, card $i$ falls into one of the 5 intervals created by the 4 Aces with equal probability, so $\mathbb{P}(I_i=1)=1/5$. Hence
\[
\mathbb{E}[X]=1+48\cdot\frac15=\frac{53}{5}.
\]

\paragraph{Generalization.}
With $m$ ordinary items and $n$ special items:
\[
\mathbb{E}[\text{position of first special}]=1+\frac{m}{m+n}.
\]

\paragraph{Intuition.}
Each ordinary card has probability $1/(m+n)$ of falling into any of the $m+n$ interval slots between special cards.

%%%%%%%%%%%%%%%%%%%%%%%%%%%%%%%%%%%%%%%%%%%%%%%%%%%%%%%%%%%%%%
\subsection{Dealing the Whole Deck}
%%%%%%%%%%%%%%%%%%%%%%%%%%%%%%%%%%%%%%%%%%%%%%%%%%%%%%%%%%%%%%

\subsubsection{Counting Poker Hands}
The total number of 5-card hands is $\binom{52}{5}$.

Below are standard counts, rewritten for clarity:

\begin{itemize}[leftmargin=1.2cm]
\item \textbf{One Pair:}
\[
\binom{13}{1}\binom{4}{2}\binom{12}{3}4^{3}.
\]

\item \textbf{Two Pairs:}
\[
\binom{13}{2}\binom{4}{2}^2 \binom{11}{1}4.
\]

\item \textbf{Three of a Kind:}
\[
\binom{13}{1}\binom{4}{3}\binom{12}{2}4^2.
\]

\item \textbf{Straight:}
\[
10\cdot(4^5-4).
\]

\item \textbf{Flush:}
\[
4\cdot\binom{13}{5}-10\cdot 4.
\]

\item \textbf{Full House:}
\[
13\binom{4}{3}\cdot 12\binom{4}{2}.
\]

\item \textbf{Four of a Kind:}
\[
13\cdot 12\cdot 4.
\]

\item \textbf{Straight Flush:}
\[
10\cdot 4 - 4.
\]

\item \textbf{Royal Flush:}
\[
4.
\]
\end{itemize}

\paragraph{Intuition.}
A straight flush is extremely rare: the probability is roughly $1$ in $65{,}000$.

\subsubsection{Probability That Each Pile Receives Exactly One Ace}

Shuffle a full deck of 52 cards uniformly and deal into four piles of 13.
Track the locations of the four Aces among the 52 positions.

\paragraph{First Ace.}
The first Ace can appear in any position without restriction:
\[
\Pr(A_1) = 1.
\]

\paragraph{Second Ace.}
After the first Ace is placed, one pile contains an Ace and must not receive another.
There are \(51\) remaining positions, of which \(39\) lie in the other three piles:
\[
\Pr(A_2 \text{ in a new pile} \mid A_1) = \frac{39}{51}.
\]

\paragraph{Third Ace.}
Two piles now contain one Ace each.
Among the \(50\) remaining positions, \(26\) lie in the two piles that do not yet contain an Ace:
\[
\Pr(A_3 \text{ in a new pile} \mid A_1, A_2) = \frac{26}{50}.
\]

\paragraph{Fourth Ace.}
Three piles contain one Ace each.
Among the \(49\) remaining positions, the final pile contains \(13\) available positions:
\[
\Pr(A_4 \text{ in the last empty pile} \mid A_1, A_2, A_3) = \frac{13}{49}.
\]

\paragraph{Final Probability.}
Multiplying the conditional probabilities gives
\[
\Pr(\text{each pile gets exactly one Ace})
= 1 \times \frac{39}{51} \times \frac{26}{50} \times \frac{13}{49}.
\]

%%%%%%%%%%%%%%%%%%%%%%%%%%%%%%%%%%%%%%%%%%%%%%%%%%%%%%%%%%%%%%
\section{Coin and Dice Models}
%%%%%%%%%%%%%%%%%%%%%%%%%%%%%%%%%%%%%%%%%%%%%%%%%%%%%%%%%%%%%%

\subsection{Geometric Distribution}
Let $X$ be the number of tosses until the first success.
\[
\mathbb{P}(X=k)=(1-p)^{k-1}p,\quad
\mathbb{E}[X]=\frac{1}{p}.
\]

\paragraph{Intuition.}
Each trial is a fresh chance to succeed; on average you need $1/p$ attempts.

\subsection{Negative Binomial Distribution}
Number of tosses to obtain $n$ successes:
\[
\mathbb{P}(X=k)=\binom{k-1}{n-1}p^n(1-p)^{k-n},\qquad
\mathbb{E}[X]=\frac{n}{p}.
\]

\subsection{Expected Time to $n$ Consecutive Heads}

We assume a fair coin, i.e.\ each toss is Heads (H) or Tails (T) with probability $1/2$.

Let $g(k)$ denote the expected additional number of tosses needed to obtain
$n$ consecutive Heads, given that we currently have a run of $k$ consecutive Heads.
Thus
\[
g(k) = \mathbb{E}[\text{time to reach $n$ Heads in a row} \mid \text{current run length } k],
\]
for $k=0,1,\dots,n$, and in particular we want $g(0)$.
Once we already have $n$ consecutive Heads, we are done, so
\[
g(n)=0.
\]

\paragraph{Recurrence.}
For $0 \le k < n$, consider the next coin toss starting from state $k$:
\begin{itemize}
  \item With probability $\tfrac{1}{2}$ we toss H, so the run of Heads increases
        from $k$ to $k+1$, and the expected remaining time becomes $g(k+1)$.
  \item With probability $\tfrac{1}{2}$ we toss T, so the run is broken and we return
        to state $0$, with expected remaining time $g(0)$.
\end{itemize}
In either case, we have used $1$ toss. Hence, for $k=0,1,\dots,n-1$,
\[
g(k)
= 1 + \frac{1}{2}g(k+1) + \frac{1}{2}g(0).
\tag{$\ast$}
\]

\paragraph{Difference equation.}
For $k=0,\dots,n-2$, write the recurrence \eqref{*} for $k$ and $k+1$ and subtract:
\begin{align*}
g(k) &= 1 + \tfrac{1}{2}g(k+1) + \tfrac{1}{2}g(0), \\
g(k+1) &= 1 + \tfrac{1}{2}g(k+2) + \tfrac{1}{2}g(0).
\end{align*}
Subtracting the second from the first gives
\[
g(k) - g(k+1) = \frac{1}{2}\bigl(g(k+1) - g(k+2)\bigr).
\]
Define the differences
\[
d_k := g(k) - g(k+1), \qquad k=0,1,\dots,n-1.
\]
Then the relation above becomes
\[
d_k = \frac{1}{2} d_{k+1}, \qquad k=0,1,\dots,n-2.
\]
Thus the differences form a geometric progression:
\[
d_{k+1} = 2 d_k, \qquad
d_k = 2^{k-(n-1)} d_{n-1}, \qquad k=0,1,\dots,n-1.
\]
Using $g(n)=0$, we have
\[
d_{n-1} = g(n-1) - g(n) = g(n-1),
\]
so
\[
d_k = g(k) - g(k+1)
    = 2^{k+1-n}\,g(n-1), \qquad k=0,1,\dots,n-1.
\]

\paragraph{Expressing $g(0)$ in terms of $g(n-1)$.}
Summing the differences from $k=0$ to $n-1$ gives
\[
g(0) - g(n) = \sum_{k=0}^{n-1} (g(k) - g(k+1))
           = \sum_{k=0}^{n-1} d_k
           = g(n-1) \sum_{k=0}^{n-1} 2^{k+1-n}.
\]
Since $g(n)=0$, this becomes
\[
g(0) = g(n-1) \, 2^{1-n} \sum_{k=0}^{n-1} 2^k
     = g(n-1) \, 2^{1-n} \bigl(2^n - 1\bigr)
     = g(n-1)\bigl(2 - 2^{1-n}\bigr).
\tag{$1$}
\]

On the other hand, for $k=n-1$ the recurrence \eqref{*} reads
\[
g(n-1)
= 1 + \frac{1}{2}g(n) + \frac{1}{2}g(0)
= 1 + \frac{1}{2}g(0),
\tag{$2$}
\]
since $g(n)=0$.

\paragraph{Solving for $g(0)$.}
Substitute \eqref{2} into \eqref{1}:
\[
g(0)
= \bigl(1 + \tfrac{1}{2}g(0)\bigr)\bigl(2 - 2^{1-n}\bigr).
\]
Expand:
\[
g(0)
= 2 - 2^{1-n} + g(0)\bigl(1 - 2^{-n}\bigr).
\]
Move the $g(0)$ terms to one side:
\[
g(0) - g(0)\bigl(1 - 2^{-n}\bigr)
= 2 - 2^{1-n},
\]
so
\[
g(0)\,2^{-n} = 2 - 2^{1-n}.
\]
Multiplying both sides by $2^n$ gives
\[
g(0) = 2^{n+1} - 2.
\]


%%%%%%%%%%%%%%%%%%%%%%%%%%%%%%%%%%%%%%%%%%%%%%%%%%%%%%%%%%%%%%
\section{Coupon Collector}
%%%%%%%%%%%%%%%%%%%%%%%%%%%%%%%%%%%%%%%%%%%%%%%%%%%%%%%%%%%%%%

Expected time to collect all $N$ coupon types:
\[
\mathbb{E}[T]=N H_N.
\]

Expected number of distinct types after $n$ trials:
\[
\mathbb{E}[\text{distinct}]=N\left(1-\left(1-\frac1N\right)^n\right).
\]

%%%%%%%%%%%%%%%%%%%%%%%%%%%%%%%%%%%%%%%%%%%%%%%%%%%%%%%%%%%%%%
\section{Approximations and Sequence Counts}
%%%%%%%%%%%%%%%%%%%%%%%%%%%%%%%%%%%%%%%%%%%%%%%%%%%%%%%%%%%%%%

\subsection{Occurrences of a Pattern}
Expected occurrences of a specific sequence of length $x$ in $n$ tosses:
\[
(n-x+1)\left(\frac12\right)^x.
\]

\subsection{Binomial Approximation}
\[
X\sim\text{Bin}(n,p)\approx \mathcal{N}(np,np(1-p)).
\]

%%%%%%%%%%%%%%%%%%%%%%%%%%%%%%%%%%%%%%%%%%%%%%%%%%%%%%%%%%%%%%
\section{Normal Distribution and MGF}
%%%%%%%%%%%%%%%%%%%%%%%%%%%%%%%%%%%%%%%%%%%%%%%%%%%%%%%%%%%%%%

\[
M_X(t)=\exp\!\left(\mu t+\frac12\sigma^2 t^2\right).
\]

%%%%%%%%%%%%%%%%%%%%%%%%%%%%%%%%%%%%%%%%%%%%%%%%%%%%%%%%%%%%%%
\section{Sums and Products of Random Variables}
%%%%%%%%%%%%%%%%%%%%%%%%%%%%%%%%%%%%%%%%%%%%%%%%%%%%%%%%%%%%%%

\subsection{Sum of Uniform Random Variables}
For $X_i\sim U[0,1]$,
\[
\mathbb{P}(X_1+\cdots+X_n\le1)=\frac{1}{n!}.
\]

\paragraph{Intuition.}
The region $x_1+\cdots+x_n\le 1$ in the $n$-cube is an $n$-simplex whose volume is $1/n!$.

%%%%%%%%%%%%%%%%%%%%%%%%%%%%%%%%%%%%%%%%%%%%%%%%%%%%%%%%%%%%%%
\section{Stick-Breaking Triangle Problem}
%%%%%%%%%%%%%%%%%%%%%%%%%%%%%%%%%%%%%%%%%%%%%%%%%%%%%%%%%%%%%%

Break at $X<Y$ with $X,Y\sim U[0,1]$. A triangle forms iff all pieces $<1/2$. The probability is:
\[
\frac14.
\]

%%%%%%%%%%%%%%%%%%%%%%%%%%%%%%%%%%%%%%%%%%%%%%%%%%%%%%%%%%%%%%
\section{Order Statistics}
%%%%%%%%%%%%%%%%%%%%%%%%%%%%%%%%%%%%%%%%%%%%%%%%%%%%%%%%%%%%%%

\subsection{Max and Min}
For i.i.d.\ variables with CDF $F$:
\[
\mathbb{P}(M\ge x)=(1-F(x))^n,\qquad
\mathbb{P}(Z\le x)=F(x)^n.
\]

\subsection{Correlation of Max and Min for Two Uniforms}
Let $X_1,X_2\sim U[0,1]$. Then:
\[
\mathbb{E}[\min]=\frac13,\qquad
\mathbb{E}[\max]=\frac23,\qquad
\mathbb{E}[\min\cdot \max]=\frac14.
\]

%%%%%%%%%%%%%%%%%%%%%%%%%%%%%%%%%%%%%%%%%%%%%%%%%%%%%%%%%%%%%%
\section{Transformations and Jensen's Inequality}
%%%%%%%%%%%%%%%%%%%%%%%%%%%%%%%%%%%%%%%%%%%%%%%%%%%%%%%%%%%%%%

If $Y=g(X)$ and $g$ is monotone, then
\[
f_Y(y)=f_X(g^{-1}(y))\frac{1}{|g'(g^{-1}(y))|}.
\]

Jensen:
\[
\mathbb{E}[g(X)]\ge g(\mathbb{E}[X]).
\]

%%%%%%%%%%%%%%%%%%%%%%%%%%%%%%%%%%%%%%%%%%%%%%%%%%%%%%%%%%%%%%
\section{Moments and Sampling}
%%%%%%%%%%%%%%%%%%%%%%%%%%%%%%%%%%%%%%%%%%%%%%%%%%%%%%%%%%%%%%

\[
\frac{(n-1)S^2}{\sigma^2}\sim \chi^2_{n-1}.
\]

%%%%%%%%%%%%%%%%%%%%%%%%%%%%%%%%%%%%%%%%%%%%%%%%%%%%%%%%%%%%%%
\section{Correlation, Covariance, and Portfolios}
%%%%%%%%%%%%%%%%%%%%%%%%%%%%%%%%%%%%%%%%%%%%%%%%%%%%%%%%%%%%%%

\subsection{Constructing Correlated Gaussians}
\[
X=aZ_1,\qquad
Y=b(\rho Z_1+\sqrt{1-\rho^2}\,Z_2).
\]

\subsection{Minimum-Variance Hedge Ratio}
Variance of $A-hB$:
\[
\mathrm{Var}(A-hB)=\sigma_A^2-2h\rho\sigma_A\sigma_B+h^2\sigma_B^2.
\]
Setting derivative 0 gives:
\[
h^*=\rho\frac{\sigma_A}{\sigma_B}.
\]

%%%%%%%%%%%%%%%%%%%%%%%%%%%%%%%%%%%%%%%%%%%%%%%%%%%%%%%%%%%%%%
\end{document}
